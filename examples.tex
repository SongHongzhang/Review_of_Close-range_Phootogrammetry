\documentclass[a4paper]{ctexart}
%\CTEXsetup[format={\Large\bfseries}]{section}
%\everymath{\displaystyle}
\usepackage{cleveref,float}
\usepackage{geometry,bm}
\usepackage{makecell,pifont,amsmath}
\usepackage{multirow}
\setcounter{MaxMatrixCols}{20}
\usepackage{booktabs}
\usepackage{graphicx}
\geometry{
  left=17mm,
  right=17mm,
  top=17mm, 
  bottom=17mm,
}
\usepackage{subfigure}
\usepackage{hyperref}
\hypersetup{
    colorlinks=true,
    linkcolor=black,
    filecolor=black,      
    urlcolor=black,
    citecolor=black,
}
\usepackage{amsmath}
\usepackage{ntheorem}
\theorembodyfont{\songti}
\newtheorem{example}{\hskip 2em 例}
\newtheorem*{zm}{\hskip 2em 证明}
\newtheorem*{sol}{\hskip 2em 解}
\newtheorem*{tips}{\hskip 2em 注}
\usepackage{fancyhdr}
\fancypagestyle{plain}{
\fancyhf{}
\fancyhead[RE]{} % 在偶数页的右侧显示章名,
\fancyhead[LO]{} % 在奇数页的左侧显示小节名,
\fancyhead[LE,RO]{~\thepage~} % 在偶数页的左侧,奇数页的右侧显示页码
\fancyhead[CE]{近景摄影测量复习}		
\fancyhead[CO]{近景摄影测量复习}	
\renewcommand{\headrulewidth}{0.7pt}
\renewcommand{\footrulewidth}{0pt}
}
\title{近景摄影测量复习}
\begin{document}
\setcounter{page}{1}
\pagestyle{plain}
\section*{作业题}
\begin{example}
物方空间布置有20个控制点,请列出使用单像空间后方交会解算内、外方位元素以及畸变系数的误差方程式,指出误差方程式、未知数、多余观测数的个数。其中畸变差模型为
$$
\begin{cases}
	\Delta x=\left( x-x_0 \right) \left[ r^2k_1+r^4k_2 \right]\\
	\Delta y=\left( y-y_0 \right) \left[ r^2k_1+r^4k_2 \right]\\
\end{cases}
$$
\end{example}
\begin{sol}
误差方程式为
$$
\boldsymbol{V}=\boldsymbol{At}+\boldsymbol{CX}_2+\boldsymbol{DX}_{ad}-\boldsymbol{L}
$$
其中,
$$
\boldsymbol{V}=\left[ \begin{matrix}
	v_x&		v_y\\
\end{matrix} \right] ^{\mathrm{T}}
$$
$$
\boldsymbol{A}=\left[ \begin{matrix}
	a_{11}&		a_{12}&		a_{13}&		a_{14}&		a_{15}&		a_{16}\\
	a_{21}&		a_{22}&		a_{23}&		a_{24}&		a_{25}&		a_{26}\\
\end{matrix} \right] 
$$
$$
\boldsymbol{t}=\left[ \begin{matrix}
	\Delta X_S&		\Delta Y_S&		\Delta Z_S&		\Delta \varphi&		\Delta \omega&		\Delta \kappa\\
\end{matrix} \right] ^{\mathrm{T}}
$$
$$
\boldsymbol{C}=\left[ \begin{matrix}
	a_{17}&		a_{18}&		a_{19}\\
	a_{27}&		a_{28}&		a_{29}\\
\end{matrix} \right] 
$$
$$
\boldsymbol{D}=\left[ \begin{matrix}
	\left( x-x_0 \right) r^2&		\left( x-x_0 \right) r^4\\
	\left( y-y_0 \right) r^2&		\left( y-y_0 \right) r^4\\
\end{matrix} \right] 
$$
$$
\boldsymbol{X}_{ad}=\left[ \begin{matrix}
	k_1&		k_2\\
\end{matrix} \right] ^{\mathrm{T}}
$$
$$
\boldsymbol{L}=\left[ \begin{matrix}
	x-\left( x \right)&		y-\left( y \right)\\
\end{matrix} \right] ^{\mathrm{T}}
$$

误差方程式的个数:40 个

未知数:$3+6+2=11$ 个

多余观测数:$40-11=29$ 个
\end{sol}
\begin{tips}
对于 $\bm{D}$,要保证 $\bm{DX}_{ad}$ 乘出来是畸变差模型的那几项就可以了。
\end{tips}
\newpage
\begin{example}
物方空间布置有 10 个控制点,列出解求 $l_i$ 系数精确解及畸变系数 $k_1$、$k_2$ 的误差方程式。其中畸变差模型为
$$
\begin{cases}
	\Delta x=\left( x-x_0 \right) \left[ r^2k_1+r^4k_2 \right]\\
	\Delta y=\left( y-y_0 \right) \left[ r^2k_1+r^4k_2 \right]\\
\end{cases}
$$
\end{example}
\begin{sol}
当有多余观测值,当像点坐标观测值改正数为 $(v_x,v_y)$,像点坐标的非线性改正为 $(\Delta x,\Delta y)$,直接线性变换取如下形式:
$$
\begin{cases}
	\left( x+v_x \right) +\Delta x+\frac{l_1X+l_2Y+l_3Z+l_4}{l_9X+l_{10}Y+l_{11}Z+1}=0\\
	\left( y+v_y \right) +\Delta y+\frac{l_5X+l_6Y+l_7Z+l_8}{l_9X+l_{10}Y+l_{11}Z+1}=0\\
\end{cases}
$$

记 $A=l_9X+l_{10}Y+l_{11}Z+1$,代入像点坐标的非线性改正得到
$$
\begin{cases}
	A\left( x+v_x \right) +A\left( x-x_0 \right) \left[ r^2k_1+r^4k_2 \right] +l_1X+l_2Y+l_3Z+l_4=0\\
	A\left( y+v_y \right) +A\left( y-y_0 \right) \left[ r^2k_1+r^4k_2 \right] +l_5X+l_6Y+l_7Z+l_8=0\\
\end{cases}
$$
整理得到 
$$
\begin{cases}
	v_x=-\frac{1}{A}\left[ A\left( x-x_0 \right) \left( r^2k_1+r^4k_2 \right) +l_1X+l_2Y+l_3Z+l_4+xXl_9+xYl_{10}+xZl_{11}+x \right]\\
	v_y=-\frac{1}{A}\left[ A\left( y-y_0 \right) \left( r^2k_1+r^4k_2 \right) +l_5X+l_5Y+l_7Z+l_8+yXl_9+yYl_{10}+yZl_{11}+y \right]\\
\end{cases}
$$

从而误差方程为
$$
\bm{V=ML-W}
$$
其中,
$$
\boldsymbol{V}=\left[ \begin{matrix}
	v_x&		v_y\\
\end{matrix} \right] ^{\mathrm{T}}
$$
$$
\boldsymbol{M}=-\left[ \begin{matrix}
	\frac{X}{A}&		\frac{Y}{A}&		\frac{Z}{A}&		\frac{1}{A}&		0&		0&		0&		0&		\frac{xX}{A}&		\frac{xY}{A}&		\frac{xZ}{A}&		\left( x-x_0 \right) r^2&		\left( x-x_0 \right) r^4\\
	0&		0&		0&		0&		\frac{X}{A}&		\frac{Y}{A}&		\frac{Z}{A}&		\frac{1}{A}&		\frac{xX}{A}&		\frac{xY}{A}&		\frac{xZ}{A}&		\left( x-x_0 \right) r^2&		\left( x-x_0 \right) r^4\\
\end{matrix} \right] 
$$
$$
\boldsymbol{L}=\left[ \begin{matrix}
	l_1&		l_2&		l_3&		l_4&		l_5&		l_6&		l_7&		l_8&		l_9&		l_{10}&		l_{11}&		k_1&		k_2\\
\end{matrix} \right] ^{\mathrm{T}}
$$
$$
\boldsymbol{W}=\left[ \begin{matrix}
	-\frac{x}{A}&		-\frac{y}{A}\\
\end{matrix} \right] ^{\mathrm{T}}
$$
\end{sol}
\begin{tips}
1. 若只需要求解 $k_1$,则只代入含 $k_1$ 的非线性改正项即可。

2. 观测值:像点坐标(20 个)

未知数:DLT 参数、畸变系数($11+2=13$ 个)
\end{tips}
\newpage
\section*{真题}
\begin{example}
使用 Kodak Professional DCS Pro SLR/n 型数码相机对某目标物摄影,该目标物在摄影方向的纵深为 2.5m,镜头焦距为 50mm,安置的光圈号数为 11,若取模糊圈直径为 20mm,相机调焦在目标物的中心,距离为 4m,请计算超焦距,并判断能否获得清晰影像。
\end{example}
\begin{sol}
超焦距 
$$
H=\frac{F^2}{k\cdot E}=\frac{50^2}{11\cdot 20}=11.4\mathrm{mm}
$$

调焦距 $D=4$m

前景深
$$
D_1=\frac{H\cdot D}{H+D}=\frac{11.4\times 4000}{11.4+4000}=11.37\text{mm}
$$

后景深
$$
D_2=\frac{H\cdot D}{H-D}=\frac{11.4\times 4000}{11.4-4000}=-11.43\text{mm}
$$

后景深为负数,说明能获得清晰影像的范围只有 $0\sim 11.37$mm,故不能获得清晰影像
\end{sol}
\begin{example}
由两台 Hasselblad 555ELD 型数码相机组成的立体摄影测量系统,使用 40mm 镜头,两相机间的距离为 1.5m,现对某试验模型按正直摄影方式拍摄立体影像对,同名点的影像坐标按单片方式量测,量测精度为 $\pm$2 像素,像素大小为 7.07 mm,若要求摄影方向的测量精度优于 $\pm$3.0mm,请估算最长摄影距离应设置为多少?
\end{example}
\begin{sol}
$m_x=\pm 2\times 7.07=14.14$mm

$m_Z=\pm \sqrt{2}\cdot \dfrac{H^2}{f\cdot B}m_x=\pm \dfrac{14.14\sqrt{2}}{40\times 1500}H^2=\pm 3\mathrm{mm}$,从而 $H=\sqrt{\dfrac{3\times 40\times 1500}{14.14\times \sqrt{2}}}=94.9\text{mm}=9.49\text{cm}$
\end{sol}
\begin{example}
\label{ex3}
就共线条件方程,回答下列问题

(1)写出以像点坐标为观测值的误差方程一般式,并写出各符号所代表的含义

(2)为测量某矿体模型的变形情况,在其表面贴敷215个标志点作为变形监测点,另外在边框上布置 20 个分布合理的标志点用全站仪测量其三维坐标作为控制点。使用数码相机在 9 个摄站对矿体模型摄影(不调焦),重叠度为100\%。若物方控制点坐标作为真值,实地不测外方位元素,列出以内、外方位元素及待定点物方坐标为未知数的光束法平差的误差方程式。指出哪类参数是观测值?误差方程式的个数是多少?多余观测数是多少?

(3)什么是近景摄影测量的光线束法平差?与空间后方交会-前方交会解法的区别是什么?
\end{example}
\begin{sol}
(1)\begin{align*}
\left[ \begin{array}{c}
	v_x\\
	v_y\\
\end{array} \right] =&\left[ \begin{matrix}
	\frac{\partial x}{\partial X_S}&		\frac{\partial x}{\partial Y_S}&		\frac{\partial x}{\partial Z_S}&		\frac{\partial x}{\partial \varphi}&		\frac{\partial x}{\partial \omega}&		\frac{\partial x}{\partial \kappa}\\
	\frac{\partial y}{\partial X_S}&		\frac{\partial y}{\partial Y_S}&		\frac{\partial y}{\partial Z_S}&		\frac{\partial y}{\partial \varphi}&		\frac{\partial y}{\partial \omega}&		\frac{\partial y}{\partial \kappa}\\
\end{matrix} \right] \left[ \begin{array}{c}
	\Delta X_S\\
	\Delta Y_S\\
	\Delta Z_S\\
	\Delta \varphi\\
	\Delta \omega\\
	\Delta \kappa\\
\end{array} \right] +\left[ \begin{matrix}
	\frac{\partial x}{\partial X}&		\frac{\partial x}{\partial Y}&		\frac{\partial x}{\partial Z}\\
	\frac{\partial y}{\partial X}&		\frac{\partial y}{\partial Y}&		\frac{\partial y}{\partial Z}\\
\end{matrix} \right] \left[ \begin{array}{c}
	\Delta X\\
	\Delta Y\\
	\Delta Z\\
\end{array} \right] \\
&+\left[ \begin{matrix}
	\frac{\partial x}{\partial f}&		\frac{\partial x}{\partial x_0}&		\frac{\partial x}{\partial y_0}\\
	\frac{\partial y}{\partial f}&		\frac{\partial y}{\partial x_0}&		\frac{\partial y}{\partial y_0}\\
\end{matrix} \right] \left[ \begin{array}{c}
	\Delta f\\
	\Delta x_0\\
	\Delta y_0\\
\end{array} \right] -\left[ \begin{array}{c}
	x-\left( x \right)\\
	y-\left( y \right)\\
\end{array} \right] 
\end{align*}
其中,$v_x$、$v_y$ 为像点坐标改正数,$(x,y)$ 为像点坐标观测值,$(X,Y,Z)$ 为控制点的物方空间坐标,$(X_S,Y_S,Z_S,\varphi,\omega,\kappa)$ 为外方位元素,$(\Delta X_S,\Delta Y_S,\Delta Z_S,\Delta \varphi,\Delta \omega,\Delta \kappa)$ 为外方位元素改正数,$(x_0,y_0,f)$ 为内方位元素,$(\Delta x_0,\Delta y_0,\Delta f)$ 为内方位元素改正数,$(x)$、$(y)$ 是前一次迭代运算结果近似值。

(2)$$
\begin{cases}
	\boldsymbol{V}_c=\boldsymbol{A}_c\boldsymbol{t}+\boldsymbol{C}_c\boldsymbol{X}_2+\boldsymbol{D}_c\boldsymbol{X}_{ad}-\boldsymbol{L}_c\,\,, \boldsymbol{P}_c\\
	\boldsymbol{V}_u=\boldsymbol{A}_u\boldsymbol{t}+\boldsymbol{B}_u\boldsymbol{X}_u+\boldsymbol{C}_u\boldsymbol{X}_2+\boldsymbol{D}_u\boldsymbol{X}_{ad}-\boldsymbol{L}_u\,\,, \boldsymbol{P}_u\\
\end{cases}
$$

观测值:待定点像点坐标,控制点像点坐标

误差方程式的个数:$9\times (20\times 2+215\times 2)=4230$(9 张像片,每个点 2 个误差方程)

必要观测数:$215\times 3+9\times 6+3\times 9=726$

多余观测数:$4230-702=3504$

(3)把控制点的像点坐标、待定点的像点坐标以 至其它内外业量测数据的一部分或全部均视作观 测值,按共线条件方程整体地、同时地解算它们的最或是值和待定点的物方空间坐标的解算方法。

与空间后交—前交的区别:\ding{172}空间后方交会-空间前方交会解法分步解求,光线束法为整体解算;\ding{173}空间后交—前交解法中待定点的像点坐标对外方位元素的确定不起作用;光线束法中,待定点的像点坐标对外方位元素的确定有很大影响。
\end{sol}
\begin{example}
就直接线性变换解法,回答下列问题

(1)写出三维直接线性变换解法的基本关系式,指出各符号的含义。

(2)条件同例 \ref{ex3},列出计算第 5 张影像的 $l_i$ 系数精确值的误差方程式。畸变差模型取
$$
\begin{cases}
	\Delta x=\left( x-x_0 \right) \left[ r^2k_1+r^4k_2 \right]\\
	\Delta y=\left( y-y_0 \right) \left[ r^2k_1+r^4k_2 \right]\\
\end{cases}
$$
\end{example}
\begin{sol}
(1)$$
\begin{cases}
	x+\frac{l_1X+l_2Y+l_3Z+l_4}{l_9X+l_{10}Y+l_{11}Z+1}=0\\
	y+\frac{l_5X+l_6Y+l_7Z+l_8}{l_9X+l_{10}Y+l_{11}Z+1}=0\\
\end{cases}
$$
其中,$l_i$ 为直接线性变换系数,$(x,y)$ 为像点坐标,$(X,Y,Z)$ 为物方空间坐标。

(2)当有多余观测值,当像点坐标观测值改正数为 $(v_x,v_y)$,像点坐标的非线性改正为 $(\Delta x,\Delta y)$,直接线性变换取如下形式:
$$
\begin{cases}
	\left( x+v_x \right) +\Delta x+\frac{l_1X+l_2Y+l_3Z+l_4}{l_9X+l_{10}Y+l_{11}Z+1}=0\\
	\left( y+v_y \right) +\Delta y+\frac{l_5X+l_6Y+l_7Z+l_8}{l_9X+l_{10}Y+l_{11}Z+1}=0\\
\end{cases}
$$

记 $A=l_9X+l_{10}Y+l_{11}Z+1$,代入像点坐标的非线性改正得到
$$
\begin{cases}
	A\left( x+v_x \right) +A\left( x-x_0 \right) \left[ r^2k_1+r^4k_2 \right] +l_1X+l_2Y+l_3Z+l_4=0\\
	A\left( y+v_y \right) +A\left( y-y_0 \right) \left[ r^2k_1+r^4k_2 \right] +l_5X+l_6Y+l_7Z+l_8=0\\
\end{cases}
$$
整理得到 
$$
\begin{cases}
	v_x=-\frac{1}{A}\left[ A\left( x-x_0 \right) \left( r^2k_1+r^4k_2 \right) +l_1X+l_2Y+l_3Z+l_4+xXl_9+xYl_{10}+xZl_{11}+x \right]\\
	v_y=-\frac{1}{A}\left[ A\left( y-y_0 \right) \left( r^2k_1+r^4k_2 \right) +l_5X+l_5Y+l_7Z+l_8+yXl_9+yYl_{10}+yZl_{11}+y \right]\\
\end{cases}
$$

从而误差方程为
$$
\bm{V=ML-W}
$$
其中,
$$
\boldsymbol{V}=\left[ \begin{matrix}
	v_x&		v_y\\
\end{matrix} \right] ^{\mathrm{T}}
$$
$$
\boldsymbol{M}=-\left[ \begin{matrix}
	\frac{X}{A}&		\frac{Y}{A}&		\frac{Z}{A}&		\frac{1}{A}&		0&		0&		0&		0&		\frac{xX}{A}&		\frac{xY}{A}&		\frac{xZ}{A}&		\left( x-x_0 \right) r^2&		\left( x-x_0 \right) r^4\\
	0&		0&		0&		0&		\frac{X}{A}&		\frac{Y}{A}&		\frac{Z}{A}&		\frac{1}{A}&		\frac{xX}{A}&		\frac{xY}{A}&		\frac{xZ}{A}&		\left( x-x_0 \right) r^2&		\left( x-x_0 \right) r^4\\
\end{matrix} \right] 
$$
$$
\boldsymbol{L}=\left[ \begin{matrix}
	l_1&		l_2&		l_3&		l_4&		l_5&		l_6&		l_7&		l_8&		l_9&		l_{10}&		l_{11}&		k_1&		k_2\\
\end{matrix} \right] ^{\mathrm{T}}
$$
$$
\boldsymbol{W}=\left[ \begin{matrix}
	-\frac{x}{A}&		-\frac{y}{A}\\
\end{matrix} \right] ^{\mathrm{T}}
$$
\end{sol}
\end{document}